\documentclass[10pt,a4paper]{book}
\usepackage[utf8]{inputenc}
\usepackage{amsmath}
\usepackage{amsfonts}
\usepackage{amssymb}
\begin{document}
\emph{This is emphasis}
\textbf{This is bold}
\textit{This is in italics}
\emph{This is again emphasis}
\begin{tiny}
This is in tiny text mode.
\end{tiny}
\begin{scriptsize}
This is in scriptsize text mode.
\end{scriptsize}
\begin{footnotesize}
This is the size of the footnote.
\end{footnotesize}
\begin{small}
This is small size font.
\end{small}
\begin{large}
This is the large size font.
\end{large}
\begin{Large}
This is also the Large size font but much bigger than before.
\end{Large}
\begin{LARGE}
This is also the LARGE size font but much much bigger than before.
\end{LARGE}
\begin{huge}
This is the huge size font.
\end{huge}
\begin{Huge}
This is the Huge size font and is much bigger.
\end{Huge}
\begin{flushleft}
This text is left left flushed using flushleft.
\end{flushleft}

\begin{center}
This is text in the center done using center.
\end{center}

\begin{flushright}
This text is flushed to the right using flushright.  
\end{flushright}
The symbol \\ (which you will not be able to see as it is a latex command) can also be used to move to the newline and it breaks the line right where it is and can only be used in the paragraph mode.

$cos(x)$\\
Now we will end this demo and discuss various symbols and porblems related to them in the next demo.

\end{document}